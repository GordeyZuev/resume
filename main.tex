\documentclass[letterpaper,11pt]{article}
\usepackage{latexsym}

\usepackage[empty]{fullpage}
\usepackage{titlesec}
\usepackage{marvosym}
\usepackage[dvipsnames]{xcolor}
\usepackage{verbatim}
\usepackage{enumitem}
\usepackage[hidelinks]{hyperref}
\usepackage{fancyhdr}
\usepackage[english, russian]{babel}
\usepackage{tabularx}
% \input{glyphtounicode}  % Commented out - file not found

\usepackage[T2A]{fontenc}
\usepackage[utf8]{inputenc}
\usepackage[russian,english]{babel}


\pagestyle{fancy}
\fancyhf{}
\fancyfoot{}
\renewcommand{\headrulewidth}{0pt}
\renewcommand{\footrulewidth}{0pt}


% Adjust margins
\addtolength{\oddsidemargin}{-0.5in}
\addtolength{\evensidemargin}{-0.5in}
\addtolength{\textwidth}{1in}
\addtolength{\topmargin}{-.5in}
\addtolength{\textheight}{1.0in}

\urlstyle{same}

\raggedbottom
\raggedright
\setlength{\tabcolsep}{0in}

% Sections formatting
\titleformat{\section}{
  \vspace{-4pt}\scshape\raggedright\large
}{}{0em}{}[\color{gray}\titlerule \vspace{-5pt}]

% Ensure that generate pdf is machine readable/ATS parsable
\pdfgentounicode=1

%-------------------------
% Custom commands
\newcommand{\resumeItem}[1]{
  \item\small{
    {#1 \vspace{-2pt}}
  }
}

\newcommand{\resumeSubheading}[4]{
  \vspace{-2pt}\item
    \begin{tabular*}{0.97\textwidth}[t]{l@{\extracolsep{\fill}}r}
      \textbf{#1} & #2 \\
      \textit{\small#3} & \textit{\small #4} \\
    \end{tabular*}\vspace{-7pt}
}

\newcommand{\resumeSubSubheading}[2]{
    \item
    \begin{tabular*}{0.97\textwidth}{l@{\extracolsep{\fill}}r}
      \textit{\small#1} & \textit{\small #2} \\
    \end{tabular*}\vspace{-7pt}
}

\newcommand{\resumeProjectHeading}[2]{
    \item
    \begin{tabular*}{0.97\textwidth}{l@{\extracolsep{\fill}}r}
      \small#1 & #2 \\
    \end{tabular*}\vspace{-7pt}
}

\newcommand{\resumeSubItem}[1]{\resumeItem{#1}\vspace{-4pt}}

\renewcommand\labelitemii{$\vcenter{\hbox{\tiny$\bullet$}}$}

\newcommand{\resumeSubHeadingListStart}{\begin{itemize}[leftmargin=0.15in, label={}]}
\newcommand{\resumeSubHeadingListEnd}{\end{itemize}}
\newcommand{\resumeItemListStart}{\begin{itemize}}
\newcommand{\resumeItemListEnd}{\end{itemize}\vspace{-5pt}}



\setlength{\footskip}{4.1pt}
\begin{document}

 %-----------HEADING-----------
\begin{center}
    \textsc{\Huge{\textbf{Гордей Зуев}}} \\ \vspace{1pt}
    
    \href{https://t.me/gordeyzuev}{\textcolor{blue!50!black}{Telegram}} $|$
    \href{mailto:gordey.zuev@gmail.com}{\textcolor{blue!50!black}{gordey.zuev@gmail.com}} $|$
    \href{https://github.com/GordeyZuev}{\textcolor{blue!50!black}{GitHub}} $|$
    \href{https://www.linkedin.com/in/gordey-zuev/}{\textcolor{blue!50!black}{LinkedIn}}
\end{center}


 %-----------SHORT BIO-----------
\section{Краткая информация}
 \begin{itemize}[leftmargin=0.15in, label={}]
    \small{\item{
     \textbf{Python Backend-разработчик} с опытом работы над высоконагруженными системами и сервисами мониторинга данных. Специализируюсь на создании масштабируемых бэкенд-решений с использованием \textit{FastAPI}, \textit{PostgreSQL}, \textit{Redis} и современных архитектурных подходов. Обладаю глубокими знаниями в области оптимизации производительности, рефакторинга схем БД и проектирования API. Преподавательский опыт развил навыки менторства и работы в команде.
    }}
 \end{itemize}


%-----------SKILLS-----------
 \section{Навыки}
\begin{itemize}[leftmargin=0.15in, label={}, nosep]
    \small
    \item \textbf{Backend-разработка:} 
          Python, FastAPI, asyncio, REST API, WebSockets, APScheduler
    
    \item \textbf{Базы данных:} 
          PostgreSQL (\textit{SQLAlchemy, asyncpg}), 
          Redis, 
          MongoDB
    
    \item \textbf{Парсинг / Автоматизация:} 
          Selenium, httpx, BeautifulSoup, автоматизация браузера
    
    \item \textbf{DevOps / Инфраструктура:} 
          Docker, CI/CD (\textit{GitHub Actions}), Kafka, мониторинг (\textit{Sentry})
    
    \item \textbf{Тестирование:} 
          pytest, Locust
    
    \item \textbf{Дополнительно:} 
           C++ (\textit{средний}), Go (\textit{начальный})
    
    \item \textbf{Языки:} 
          русский (\textit{носитель}),
          английский (\textit{B2+, техническая документация})

    \item \textbf{Soft-Skills:}
        Преподавание, 
        менторство,
        публичные выступления
    
\end{itemize}

%-----------EXPERIENCE-----------
\section{Опыт работы}
  \resumeSubHeadingListStart

    \resumeSubheading
    {\textbf{Yandex - «DataContracts»}}{Июль 25' -- Ноябрь 25'}
    {Backend-разработчик Python $|$ Строгая типизация, FastAPI, PostgreSQL}{Москва, Россия}
    \resumeItemListStart

      \resumeItem{\textbf{Рефакторинг схемы БД:} Провёл рефакторинг схемы базы данных \textit{PostgreSQL}, что улучшило масштабируемость системы и уменьшило время выполнения ключевых запросов.}

      \resumeItem{\textbf{Механизм управления статусами:} Разработал и внедрил механизм ручного управления статусами контрактов, что позволило командам-потребителям данных самостоятельно разрешать инциденты.}
      
      \resumeItem{\textbf{Комплексные правила мониторинга:} Добавил поддержку комплексных правил мониторинга из нескольких источников данных, увеличив гибкость платформы для сложных бизнес-сценариев.}
      
      \resumeItem{\textbf{Система подсказок:} Спроектировал и реализовал систему подсказок (suggests) для мониторинга и эскалаций, сократив время создания новых контрактов.}

    \resumeItemListEnd

    \resumeSubheading
      {\textbf{Высшая Школа Экономики}}{Сен 23' -- Наст. время}
      {Образовательная деятельность}{Москва, Россия}
      \resumeItemListStart
      
        \resumeItem{\textbf{Преподавание:} \textbf{Более 860 студентов} в университете и лицее.}
        
        \resumeItem{\textbf{Кураторство и автоматизация:} Разработал систему управления обучением (Python, Notion API). Обучил ML-модель для прогнозирования отчислений. \textbf{Снижение отчислений на $\sim 47\%$}}
        
    \resumeItemListEnd

\resumeSubHeadingListEnd

%-----------PROJECTS & RESEARCH-----------
\section{Проекты и исследования}
    \resumeSubHeadingListStart

%-----------PROJECT 1-----------
    \resumeProjectHeading
    {\href{https://github.com/GordeyZuev/hse_bot_ai_masters}{\textcolor{blue!50!black}{\textbf{HSE AI Deadlines Bot}}} $|$ \emph{Python, uv, ruff, PostgreSQL, Telegram / Google API}}{Авг 25' -- Окт 25'}
    \resumeItemListStart
        \resumeItem{Разработал \textbf{Telegram-бота} с ежедневной синхронизацией данных из \textit{Google Sheets API} и автоматическим парсингом дедлайнов.}
        \resumeItem{Реализовал \textbf{систему подписок} через \textit{APScheduler} и \textbf{отказоустойчивую архитектуру с ретраями и fallback} на локальную БД.}
        \resumeItem{\underline{\textbf{Итог:}} Стабильный бот с 500+ активными пользователями, \textbf{99.8\%} uptime и автоматическим восстановлением после сбоев}
    \resumeItemListEnd

%-----------PROJECT 2-----------
    \resumeProjectHeading
    {\href{https://github.com/GordeyZuev/RestoMaps-Analytics}{\textcolor{blue!50!black}{\textbf{RestoMaps-Analytics}}} $|$ \emph{Python, Streamlit, Notion / Yandex.Maps API, Selenium, Sentry}}{Июль 25' -- Окт 25'}
    \resumeItemListStart
        \resumeItem{Создал \textbf{систему анализа ресторанных данных} с интеграцией \textit{Notion API} и \textit{Yandex Maps} для агрегации отзывов и оценок.}
        \resumeItem{Реализовал \textbf{автоматический парсинг отзывов} через \textit{Selenium} и анализ тональности.}
        \resumeItem{Разработал \textbf{интерактивную визуализацию} на \textit{Streamlit} с картами \textit{(Folium)} и аналитическими дашбордами \textit{(Plotly)}.}
        \resumeItem{\underline{\textbf{Итог:}} Комплексное решение для анализа ресторанов из личных заметок с умными фильтрами и визуализацией данных на карте}
    \resumeItemListEnd

%-----------PROJECT 3-----------
    \resumeProjectHeading
    {\href{https://github.com/GordeyZuev/ZoomUploader}{\textcolor{blue!50!black}{\textbf{ZoomUploader}}} $|$ \emph{Python, uv, ruff, FFmpeg, Zoom / YouTube / VK API}}{Окт 25'}
    \resumeItemListStart
        \resumeItem{Разработал \textbf{автоматизированную систему} для загрузки, обработки и публикации записей Zoom на \textit{YouTube} и \textit{VK}.}
        \resumeItem{Реализовал \textbf{пайплайн обработки видео} через \textit{FFmpeg} с \textbf{многопоточностью и асинхронной обработкой} нескольких видео одновременно.}
        \resumeItem{Интегрировал \textbf{многоаккаунтную поддержку Zoom API} с автоматической синхронизацией записей, метаданными и конфигурационными шаблонами.}
        \resumeItem{\underline{\textbf{Итог:}} Высокопроизводительная система автоматизации видео-контента с \textbf{параллельной обработкой} и поддержкой множественных аккаунтов}
    \resumeItemListEnd

%-----------PROJECT 4-----------
    \resumeProjectHeading
    {\textbf{Автоматическая бронь билетов} $|$ \emph{FastAPI, PostgreSQL, Redis}}{Ноя 24' -- Июнь 25'}
    \resumeItemListStart
        \resumeItem{Разработал \textbf{API-спецификации} и спроектировал схему БД с оптимизированными индексами.}
        \resumeItem{Реализовал \textbf{логику кэширования} через \textit{Redis} для снижения нагрузки на БД.}
        \resumeItem{\underline{\textbf{Итог:}} Отказоустойчивая система бронирования с кэшированием, готовая к высоким нагрузкам}
    \resumeItemListEnd

%-----------PROJECT 5-----------
    \resumeProjectHeading
    {\textbf{Football Assistant} $|$ \emph{Flutter, PostgreSQL, WebSockets}}{Июл 22' -- Май 23'}
    \resumeItemListStart
        \resumeItem{Разработал \textbf{кроссплатформенное приложение} \textit{(Flutter/Dart)} с backend и отслеживанием матчей в реальном времени.}
        \resumeItem{Интегрировал \textbf{WebSockets} для live-обновлений и провел тестирование \textit{(PyTest, Flutter test)}.}
        \resumeItem{\underline{\textbf{Итог:}} Полнофункциональное приложение для проведения и анализа футбольных матчей с real-time обновлениями}
    \resumeItemListEnd

%-----------PROJECT 6-----------
    \resumeProjectHeading
    {\textbf{ML-модель прогнозирования отчислений} $|$ \emph{Python, Scikit-learn, pandas}}{Ноя 24' -- Мар 25'}
    \resumeItemListStart
        \resumeItem{Обучил \textbf{ML-модель} на учебных и неучебных данных с результатами \textit{ROC-AUC (0.91)} и \textit{F1-score (0.84)}.}
        \resumeItem{Автоматизировал обработку данных из образовательной системы через \textit{SQLAlchemy}.}
        \resumeItem{\underline{\textbf{Итог:}} Система раннего предупреждения, снизившая уровень отчислений студентов.}
    \resumeItemListEnd

\resumeSubHeadingListEnd

%-----------EDUCATION-----------
\section{Образование}
  \resumeSubHeadingListStart
  
    \resumeSubheading
      {Высшая Школа Экономики}{Сен 22' -- Наст. время}
      {Бакалавриат - \textbf{Прикладная математика и информатика}}{Москва, Россия}

    % \resumeSubheading
    %   {НИУ ВШЭ (Высшая Школа Экономики)}{Сен 26' -- Наст. время}
    %   {Магистратура - _}{_}

\resumeSubHeadingListEnd

%-----------HONORS AND AWARDS-----------
\section{Награды и достижения}
  \resumeSubHeadingListStart
  
    \resumeSubheading
      {Победитель всероссийских олимпиад по математике}{Апр 22'}
      {«Курчатов», «Физтех» (МФТИ)}{Москва, Россия}

    \resumeSubheading
      {Награда \href{https://cs.hse.ru/news/874256115.html}{\textcolor{blue!50!black}{«Молодой предприниматель факультета»}}}{Ноя 23'}
      {Лучшая идея среди 20+ студенческих стартапов.}{Высшая школа экономики}

     \resumeSubheading
      {Благодарность от факультета}{Дек 24'}
      {Отмечен за вклад в развитие образовательной программы.}{Высшая школа экономики}
      
\resumeSubHeadingListEnd

\end{document}
\documentclass[letterpaper,11pt]{article}

\usepackage{latexsym}
\usepackage[empty]{fullpage}
\usepackage{titlesec}
\usepackage{marvosym}
\usepackage[dvipsnames]{xcolor}
\usepackage{verbatim}
\usepackage{enumitem}
\usepackage[hidelinks]{hyperref}
\usepackage{fancyhdr}
\usepackage[english, russian]{babel}
\usepackage{tabularx}
\input{glyphtounicode}

\usepackage[T2A]{fontenc}
\usepackage[utf8]{inputenc}
\usepackage[russian,english]{babel}


\pagestyle{fancy}
\fancyhf{}
\fancyfoot{}
\renewcommand{\headrulewidth}{0pt}
\renewcommand{\footrulewidth}{0pt}


% Adjust margins
\addtolength{\oddsidemargin}{-0.5in}
\addtolength{\evensidemargin}{-0.5in}
\addtolength{\textwidth}{1in}
\addtolength{\topmargin}{-.5in}
\addtolength{\textheight}{1.0in}

\urlstyle{same}

\raggedbottom
\raggedright
\setlength{\tabcolsep}{0in}

% Sections formatting
\titleformat{\section}{
  \vspace{-4pt}\scshape\raggedright\large
}{}{0em}{}[\color{gray}\titlerule \vspace{-5pt}]

% Ensure that generate pdf is machine readable/ATS parsable
\pdfgentounicode=1

%-------------------------
% Custom commands
\newcommand{\resumeItem}[1]{
  \item\small{
    {#1 \vspace{-2pt}}
  }
}

\newcommand{\resumeSubheading}[4]{
  \vspace{-2pt}\item
    \begin{tabular*}{0.97\textwidth}[t]{l@{\extracolsep{\fill}}r}
      \textbf{#1} & #2 \\
      \textit{\small#3} & \textit{\small #4} \\
    \end{tabular*}\vspace{-7pt}
}

\newcommand{\resumeSubSubheading}[2]{
    \item
    \begin{tabular*}{0.97\textwidth}{l@{\extracolsep{\fill}}r}
      \textit{\small#1} & \textit{\small #2} \\
    \end{tabular*}\vspace{-7pt}
}

\newcommand{\resumeProjectHeading}[2]{
    \item
    \begin{tabular*}{0.97\textwidth}{l@{\extracolsep{\fill}}r}
      \small#1 & #2 \\
    \end{tabular*}\vspace{-7pt}
}

\newcommand{\resumeSubItem}[1]{\resumeItem{#1}\vspace{-4pt}}

\renewcommand\labelitemii{$\vcenter{\hbox{\tiny$\bullet$}}$}

\newcommand{\resumeSubHeadingListStart}{\begin{itemize}[leftmargin=0.15in, label={}]}
\newcommand{\resumeSubHeadingListEnd}{\end{itemize}}
\newcommand{\resumeItemListStart}{\begin{itemize}}
\newcommand{\resumeItemListEnd}{\end{itemize}\vspace{-5pt}}


%-------------------------------------------
%%%%%%  RESUME STARTS HERE  %%%%%%%%%%%%%%%%%%%%%%%%%%%%

\setlength{\footskip}{4.1pt}
\begin{document}

 %-----------HEADING-----------
\begin{center}
    \textsc{\Huge{\textbf{Гордей Зуев}}} \\ \vspace{1pt}
    
    \href{{https://t.me/gordeyzuev}}{\textcolor{blue!50!black}{Telegram}} $|$
    \href{mailto:gordey.zuev@gmail.com}{\textcolor{blue!50!black} {gordey.zuev@gmail.com}} $|$
    \href{https://github.com/GordeyZuev}{\textcolor{blue!50!black}{GitHub}} $|$
    \href{{https://www.linkedin.com/in/gordey-zuev/}}{\textcolor{blue!50!black}{LinkedIn}}
\end{center}


 %-----------SHORT BIO-----------
\section{Краткая информация}
 \begin{itemize}[leftmargin=0.15in, label={}]
    \small{\item{
     \textbf{Junior Python-разработчик} с опытом в прикладной математике (НИУ ВШЭ). Специализируюсь на создании эффективных бэкенд-решений. Обожаю оптимизировать всё вокруг себя, интересуюсь архитектурой распределённых систем. Преподавательский опыт (3+ года) развил навыки менторства и объяснения сложных концепций. Участвовал в разработке высоконагруженных сервисов и ML-моделей. Лауреат конкурсов для преподавателей и олимпиад по математике.
    }}
 \end{itemize}


%-----------SKILLS-----------
 \section{Навыки}
\begin{itemize}[leftmargin=0.15in, label={}, nosep]
    \small
    \item \textbf{Backend-разработка:} 
          Python, FastAPI, Flask, asyncio, REST API, WebSockets
    
    \item \textbf{Базы данных:} 
          PostgreSQL (\textit{SQLAlchemy, asyncpg}), 
          Redis, 
          MongoDB
    
    \item \textbf{DevOps \& Инфраструктура:} 
          Docker, 
          CI/CD (\textit{GitHub Actions}), 
          Kafka
    
    \item \textbf{Тестирование:} 
          pytest, Locust
    
    \item \textbf{Дополнительно:} 
           C++ (\textit{средний}), Go (\textit{начальный})
    
    \item \textbf{Языки:} 
          русский (\textit{носитель}),
          английский (\textit{B2+, техническая документация}), 
          немецкий (\textit{B1})

    \item \textbf{Soft-Skills:}
        Преподавание,
        менторство,
        публичные выступления
    
\end{itemize}

%-----------PROJECTS & RESEARCH-----------
\section{Проекты и исследования}
    \resumeSubHeadingListStart

%-----------PROJECT 1-----------
    \resumeProjectHeading
    {\textbf{Автоматическая бронь билетов} $|$ \emph{Архитектура: Python, FastAPI, PostgreSQL, Redis}}{Ноя 24` -- Июнь 25`}
    \resumeItemListStart
        \resumeItem{Разработал \textbf{API-спецификации} (\textit{OpenAPI} / \textit{Swagger}) для интеграции \textit{frontend} и \textit{backend}.}
        \resumeItem{Спроектировал  и реализовал \textbf{схему БД (\textit{PostgreSQL})} для сервиса с оптимизированными индексами.}
        \resumeItem{Реализовал \textbf{логику кэширования} билетов через \textit{Redis} для уменьшения нагрузки на БД.}
        % \resumeItem{Провел \textbf{нагрузочное тестирование} (Locust) и оптимизировал запросы, сократив latency на 25\%}
        % \resumeItem{Настроил \textbf{Docker-окружение} для тестирования и CI/CD пайплайн}
        \resumeItem{\underline{\textbf{Итог:}} Создал отказоустойчивую систему бронирования с кэшированием и продуманной схемой БД, готовую к высоким нагрузкам}
    \resumeItemListEnd


%-----------PROJECT 2-----------
      \resumeProjectHeading
          {\textbf{Приложение «Football Assistant»} $|$ \emph{Flutter (Dart), PostgreSQL, Docker, Figma}}{Июл 22' -- Май 24'}
          \resumeItemListStart
          
            \resumeItem{\textbf{Исследование и дизайн} – Проведен анализ рынка, определены пользовательские сценарии, разработана архитектура приложения \textit{(Flutter, PostgreSQL, REST API)}.}
            
            \resumeItem{\textbf{Разработка и интеграция} – Создано кроссплатформенное приложение \textit{(Flutter, Dart)}, реализован backend \textit{(FastAPI, PostgreSQL)}, интегрировано отслеживание матчей в реальном времени.}
            
            \resumeItem{\textbf{Тестирование и оптимизация} – \textit{PyTest}, \textit{Flutter test}, \textit{WebSockets} для обновлений в реальном времени.}
             \resumeItem{\underline{\textbf{Итог:}} Разработано кроссплатформенное приложение для проведения и анализа футбольных матчей любого уровня с функцией отслеживания в реальном времени.}
          \resumeItemListEnd


 %-----------PROJECT 3-----------
    \resumeProjectHeading
        {\textbf{Обучение модели для прогнозирование риска отчисления} $|$ \emph{Python, ML}}{Ноя 24` -- Мар 25'}
        \resumeItemListStart

            \resumeItem{\textbf{Сбор и обработка данных:} \textit{Python} (\textit{pandas}, \textit{numpy}) и \textit{Scikit-learn} для сбора и обработки данных. Данные - учебные и неучебные (работа, личная жизнь).} 
            \resumeItem{\textbf{Обучение модели:} Обучение нескольких моделей с результатом - \textit{ROC-AUC} (0.91) и \textit{F1-score} (0.84).} 
            \resumeItem{Реализация на \textit{Python} (\textit{pandas}, \textit{SQLAlchemy}) автоматической обработки данных из образовательной системы}
            \resumeItem{\underline{\textbf{Итоги}}: Модель помогла создать систему раннего предупреждения студентов, снизив уровень отчислений.}
        \resumeItemListEnd

\resumeSubHeadingListEnd


%-----------EXPERIENCE-----------
\section{Опыт работы}
  \resumeSubHeadingListStart

    \resumeSubheading
      {Высшая школа экономики}{Москва, Россия}
      {\textbf{Ассистент преподавателя}}{Сен 23' -- Наст. время}
      \resumeItemListStart
      
        \resumeItem{Курсы: \textbf{«Программирование на Python»} / \textbf{«Python для анализа данных»} / \textbf{«Инструменты разработки»} / \textbf{«Прикладной Python»} \textit{(магистратура)}. Суммарно \textbf{более 860 студентов}.}
        
        \resumeItem{\textbf{Проведение code-review} всех ДЗ, разработка методического материала и \textbf{проведение консультаций}. Количество несдавших снизилось в среднем на $\sim 13\%$.}
        
    \resumeItemListEnd

    \resumeSubSubheading
      {\textbf{Куратор образовательной программы}}{Янв 24' -- Наст. время}
      \resumeItemListStart
      
        \resumeItem{Разработал систему управления обучением на основе Notion, \textbf{используя Python-скрипты  и Notion API}.}

        \resumeItem{\textbf{Обучил модель} (на учебных и неучебных данных), определяющую студентов с риском отчисления.}
        
        \resumeItem{\underline{\textbf{Итог:}} Повышение доступности информации и уменьшение числа отчислившихся на $\sim 47\%$.}
        
    \resumeItemListEnd

    \resumeSubheading
      {Преподавание}{Москва, Россия}
      {\textbf{Лицей НИУ ВШЭ}: «Программирование на Python» / «Алгоритмы»}{Сен 23' -- Май 24'}
      \resumeItemListStart
      
        \resumeItem{Средняя оценка преподавания - \textbf{4.73/5.0}. Средняя оценка учеников выросла на $\sim 18\%$.}
        
        \resumeItem{\textbf{9/16} студентов \textbf{победили как минимум в одной олимпиаде} по информатике в течение года.}
    \resumeItemListEnd

    \resumeSubSubheading
      {\textbf{Репетиторство}: «Прикладной Python» / «Анализ данных»}{Май 23' -- Наст. время}
      \resumeItemListStart
        \resumeItem{Обучил \textbf{17 студентов}. Средняя оценка \href{https://profi.ru/profile/ZuyevGA2/}{\textcolor{blue!50!black}{на сервисе по поиску репетиторов}} - 5.0 / 5.0.}
    \resumeItemListEnd

\resumeSubHeadingListEnd

%-----------EDUCATION-----------
\section{Образование}
  \resumeSubHeadingListStart
  
    \resumeSubheading
      {НИУ ВШЭ (Высшая Школа Экономики)}{Москва, Россия}
      {Бакалавриат - \textbf{Прикладная математика и направление}}{Сен 22' -- Наст. время}
        \resumeItemListStart
            \resumeItem{\textbf{Направление}: Промышленное программирование, распределенные системы.}
        \resumeItemListEnd
    
    \resumeSubheading
      {Лицей НИУ ВШЭ}{Москва, Россия}         
      {Лицей с углубленным изучением \textbf{математики и информатики}}{Сен 19' -- Май 22'}
        \resumeItemListStart
            \resumeItem{\textbf{Направление}: Математические олимпиады и программирование. \textbf{GPA}: 3.94.}
        \resumeItemListEnd
  \resumeSubHeadingListEnd


%-----------EXTRA COURCES-----------
\section{Дополнительное образование}
  \resumeSubHeadingListStart
    \resumeSubheading
      {Ассистенты преподавателя (Яндекс)}{Москва, Россия}         
      {Менторство и преподавание}{Июнь 23' -- Май 24'}

    \resumeSubheading
      {Летняя школа ФКН по предпринимательству (ВШЭ)}{Москва, Россия}         
      {Изучение пути создания и развития продукта}{Лето 23' и Лето 24'}

    \resumeSubheading
      {Курсы ораторского мастерства}{Москва, Россия}         
      {I Ступень - Речь и II Ступень - Публичные выступления}{Фев 23' -- Май 23'}
      
  \resumeSubHeadingListEnd



%-----------HONORS AND AWARDS-----------
\section{Награды и достижения}
  \resumeSubHeadingListStart
  
    \resumeSubheading
      {Награда \href{https://cs.hse.ru/initiative/bestassistants}{\textcolor{blue!50!black}{«Лучший ассистент преподавателя семестра»} x3}}{Высшая школа экономики}
      {Курсы: \textbf{«Python»}, \textbf{«Анализ данных»} и \textbf{«Инструменты разработки»}}{Сен 23' - Июнь 24'}

    \resumeSubheading
      {Награда \href{https://cs.hse.ru/news/874256115.html}{\textcolor{blue!50!black}{«Молодой предприниматель факультета»}}}{Высшая школа экономики}
      {Лучшая идея среди 20+ студенческих стартапов.}{Ноя 23'}
      
    \resumeSubheading
      {Победитель олимпиады «Курчатов» по математике (Топ 0.7\%)}{Москва, Россия}
      {Вошел в топ 0.7\% из 20,000+ участников.}{Апр 22'}

    \resumeSubheading
      {Победитель олимпиады «Физтех» (МФТИ) по математике (Топ 1.3\%)}{Москва, Россия}
      {Вошел в топ 1.3\% из 15,807 участников.}{Апр 22'}

     \resumeSubheading
      {Благодарность от факультета}{Высшая школа экономики}
      {Отмечен за вклад в развитие образовательной программы.}{Дек 24'}
      
\resumeSubHeadingListEnd

\end{document}